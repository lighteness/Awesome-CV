%-------------------------------------------------------------------------------
%	SECTION TITLE
%-------------------------------------------------------------------------------
\cvsection{工作经历}


%-------------------------------------------------------------------------------
%	CONTENT
%-------------------------------------------------------------------------------
\begin{cventries}

%---------------------------------------------------------
\cventry
  {资深基础设施平台工程师} % Job title
  {Wish电商平台} % Organization
  {} % Location
  {2022/11 - 2023/8} % Date(s)
  {
    \begin{cvitems} % Description(s) of tasks/responsibilities
      \item {Kubernetes Operator模式的使用:利用crossplane生态中的provider template,为专门管理Kafka topic设计并实现了高效的operator。}
      \item {在AWS上的Kubernetes集群管理:采用蓝绿部署策略,将EKS(Amazon Elastic Kubernetes Service)在无宕机的情况下,从1.21版本升级到1.24版本。}
      \item {容器运行时的迁移:将Kubernetes的容器运行时从Docker迁移到Containerd,以符合行业标准。}
      \item {容器存储接口的迁移:从Kubernetes的内部存储插件迁移到Container Storage Interface(CSI)驱动,并确保终端用户平滑过渡。}
      \item {内部开发平台的建设:利用argocd、crossplane和OAM(kubevela)简化工作流并自动化流程,从而加快开发周期并增加业务影响力。}
    \end{cvitems}
  }
  
%---------------------------------------------------------
  \cventry
    {Kubernetes管理员 \& PaaS工程师} % Job title
    {Paypal线支付平台} % Organization
    {} % Location
    {2020/6 - 2022/10} % Date(s)
    {
      \begin{cvitems} % Description(s) of tasks/responsibilities
        \item {Kubernetes Operator模式的使用: 使用Operator pattern来管理PayPal私有数据中心内的软负载设备,从而提高了配置效率和系统的弹性。}
        \item {Kubernetes集群搭建:为了满足金融科技环境下的安全规格,设计并完成了一个专门的工具,用于Kubernetes集群的引导和扩展。}
        \item {SSO集成及部署:设计和集成了Single Sign-On(SSO)功能,并进一步扩大其在整个平台上的应用,增强了安全性和用户体验。        }
        \item {警报系统集成:无缝集成了Prometheus和Zabbix,建立了一个健壮的警报机制,确保了实时监控和快速的问题解决。 }
      \end{cvitems}
    }










%---------------------------------------------------------
  \cventry
    {Devops工程师 \& Kubernetes管理员} % Job title
    {eBay Inc. \tiny{(外包岗)}} % Organization
    {} % Location
    {2018/7 - 2020/5} % Date(s)
    {
      \begin{cvitems} % Description(s) of tasks/responsibilities
        \item {Azure上的Kubernetes集群管理:在Azure虚拟机上,实现了Kubernetes集群部署、优化和安全管理,确保持续的在线时间、最佳性能和无缝的升级。}
        \item {Rancher部署:使用Rancher,简化开发者对集群使用,降低了对容器管理的门槛 }
        \item {实时监控解决方案:使用Prometheus实施监控基础设施,为集群性能提供实时监控,并显著加快故障排查工作。        }
        \item {CI/CD集成:设计并部署基于Jenkins的CI/CD流水线,以适应日常更新和敏捷的每周发布,以满足不断变化的业务需求。        }
        \item { 集群自动化:使用Shell、Python和Ansible编写高级自动化脚本,简化集群管理,最大限度地减少手动监督的需求。       }
      \end{cvitems}
    }
 

%---------------------------------------------------------
  \cventry
    {全栈工程师} % Job title
    {eBay电商平台\tiny{(外包岗)}} % Organization
    {} % Location
    {2016/3 - 2018/6} % Date(s)
    {
      \begin{cvitems} % Description(s) of tasks/responsibilities
        \item {API网关的实现:创建了一个基于Node.js的API网关,顺利地集成了新的后端服务,同时保留了遗留系统的价值,并对业务指标进行埋点监控。 }
        \item {用户体验增强:使用ReactJS和Redux翻新了网站,导致用户报告的问题减少了30\%,并提高了用户满意度。 }
        \item {跨团队合作:与美国团队之间建立了有效的沟通渠道,以确保全球标准的对齐。  }
        \item {质量保证:进行定期的代码审查,将部署后的漏洞减少了10\%,并保持了编码的一致性。  }
        \item {技术文档:编写了全面的技术文档,以增强团队间的合作。 }
      \end{cvitems}
    }

%---------------------------------------------------------
\cventry
  {java软件工程师 } % Job title
  {爱立信 \tiny{(外包岗)}} % Organization
  {} % Location
  {2014/5 - 2016/3} % Date(s)
  {
    \begin{cvitems} % Description(s) of tasks/responsibilities
      \item {端到端的开发:积极参与eMBMS系统(多媒体广播多播服务)的软件开发任务,独立完成需求分析,到软件功能的设计与编码。 }
      \item {高效软件交付:采用持续集成方式,提高了软件质量和更快的发布时间。}
      \item {敏捷开发:成功使用Scrum方法交付了一系列的系统改进。  }
      \item {质量保证工具:编写测试模拟器,并使用开源的测试工具,对自己的产品,做详细的单元测试和集成测试,确保了产品的更高可靠性标准。}
      \item { 技术文档:编写并维护详细的技术文档,包括基于UML的设计文档和用户指南。  }
    \end{cvitems}
  }

 

%---------------------------------------------------------
\cventry
  {java软件工程师} % Job title
  {塔塔信息技术} % Organization
  {} % Location
  {2012/5 - 2014/4} % Date(s)
  {
    \begin{cvitems} % Description(s) of tasks/responsibilities
      \item {在线报价工具:在资深工程师的协助下,完成了保险领域里的保单快速报价功能,使客户能够轻松获得在线指示性报价。}
      \item {MVC模式的使用:根据特定的保险业务需求,使用MVC模式,完成了核心计算层,从而实现了更高效的数据处理。}
      \item {数据库建模:构建与业务模型对齐的持久实体,简化数据存储和检索过程。}
      \item {质量保证和文档:创建项目文档,并通过严格的测试阶段隔离和解决缺陷,确保软件质量。  }
    \end{cvitems}
  }
  

%---------------------------------------------------------
\end{cventries}
